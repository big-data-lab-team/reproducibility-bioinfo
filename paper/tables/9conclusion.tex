\begin{table}[ht]
    \centering
    \begin{tabular}{| P{15cm} || p{1cm} |}
        \hline
        \rowcolor{gray}\multicolumn{2}{|L|}{\small{Data Provenance and Sharing}}\label{tb:dataprovenance}\\
        \hline
            \begin{itemize}
                \tabitem{Report on the datasource(s) (for sequences, report on the database, the accession number and the version)}
                \tabitem{Explain the process(es) by which the final data for the dataset was produced}
                \tabitem{Share the final dataset and provide a link to downloadable version of the dataset}
            \end{itemize} &\\
        \hline \hline

        \rowcolor{gray}\multicolumn{2}{|l|}{\small{Feature Provenance and Sharing}}\label{tb:featureprovenance}\\
        \hline
        \begin{itemize}
                \tabitem{Explain the concept associated with the extracted feature }
                \tabitem{Explain the process through which the feature is extracted}
                \tabitem{For formulas, describe the associated parameters}
                \tabitem {Share the extracted feature file. If sharing the entire generated result is not feasible, then, 
                share reasonable amount of rows from the final extracted feature file}
            \end{itemize} &\\
        \hline \hline
        \rowcolor{gray}\multicolumn{2}{|l|}{\small{Model Provenance and Sharing}}\label{tb:modelprovenance}\\
        \hline
        \multicolumn{2}{|L|}{\small{Data Pre-processing}}\label{tb:preprocessing}\\ 
        \hline
            \begin{itemize}
                \tabitem{Explain the pre-processing technique concept or provide reference to an available resource}
                \tabitem{Explain the pre-processing technique details including any involved formulas and its parameter(s)}
                \tabitem{Explain how it has been applied to the original data}
                \tabitem{Share the final transformed feature set,If sharing the entire generated result is not feasible, then, 
                share reasonable amount of rows from the final transformed feature file}
            \end{itemize}&\\
        \hline
        \multicolumn{2}{|L|}{\small{Model Structure}}\label{tb:modelstructure}\\ 
        \hline
            \begin{itemize}
                \tabitem{Explain over any applied sampling technique along with process and involved parameter(s) }
                \tabitem{Describe the strategy for splitting the original data into train, validation and test subsets}
                \tabitem{Explain the deployed algorithm, considered range of hyper-parameters and the associated values for 
                obtaining the published results}
                \tabitem{If a specific optimal hyper-parameters search technique is used, provide the final deployed values 
                resulted from the process, describe the method, any involved parameter(s), the process and 
                how it has been applied to the project's data. }
                \tabitem{If dealing with multiple classes, describe the process and how the results are aggregated}
                \tabitem{For ensemble models, report on the structure, underlying models and the aggregation strategy. Also,
                for each involved model, describe the details mentioned above.}
                \tabitem{Share reasonable amount of the generated results (where possible)}
            \end{itemize}&\\
    
        \hline
        \multicolumn{2}{|L|}{\small{Model Evaluation}}\label{tb:modelevaluation}\\ 
        \hline
            \begin{itemize}
                \tabitem{Describe the choice of statistical method used for evaluation of the results, any involved formula 
                and its parameter(s)}
                \tabitem{If averaging through multiple results, describe the methodology (micro vs macro)}
                \tabitem{Define error bars (if any)}
            \end{itemize}&\\
        \hline
        \rowcolor{gray}\multicolumn{2}{|l|}{\small{Software Provenance and Sharing}}\label{tb:softwareprovenance}\\
        \hline
            \begin{itemize}
                \tabitem{For coded programs, report on programming language(s), version, libraries and any required installation instructions.}
                \tabitem{For any involved software, report on software details and a reference to the software documentation} 
                \tabitem{If your pipeline is environment-dependent, then, provide details over the underlying infrastructure 
                and how to setup the pipeline through that environment.}
            \end{itemize} &\\
        \hline
        \rowcolor{gray}\multicolumn{2}{|l|}{\small{Pipeline Provenance and Sharing}}\label{tb:pipelineprovenance}\\
        \hline
            \begin{itemize}
                \tabitem{Design and code your pipeline as a chain of independently runnable modules (i.e. a module for feature extraction) 
                in a way that the output from each phase would be the input to the next one.}
                
                \tabitem{Create a Jupyter Notebook (or any available alternative) to walk the reader through the pipeline 
                starting from the feature extraction all the way to result generation. for each step:}
                    \begin{itemize}
                        \tabitem{Explain briefly what the code does (or make a reference to the correspondent part in the paper) 
                        followed by a runnable code cell demonstrating how to run the code.}
                        \tabitem{If running the entire code for a part (i.e. feature selection) is not feasible, then leave the runnable 
                        code cell there and explain why it is not feasible to run it through the notebook. Also, explain over the expected output 
                        from this process (if available, you can load and display the process result instead)}
                \end{itemize}
                \tabitem{If a random function is used anywhere throughout the whole process, then, 
                report on random function seed and the library details}
                
                \tabitem{Report on any involved coded program, software or environment-related parameters, 
                  according to section \ref{sec:softwareProvenance} guidelines. }
                
                \tabitem{Version control and share the entire project (including the output from each phase)}
                \tabitem{If possible, setup your pipeline on a container (or a virtual machine) and share the entire virtual environment in 
                the state that the results are produced.}
                
            
            \end{itemize} &\\
        \hline

    \end{tabular}
    \captionsetup{font=small,width=12cm}
    \caption{Reproducible experiment report checklist}
    \label{tab:table3}
    
\end{table}

% \begin{table}[ht]
%     \centering
%     \begin{tabular}{| P{14cm} || p{2cm} |}
%         \hline
%         \rowcolor{gray}\multicolumn{2}{|l|}{\small{Pipeline Provenance and Sharing}}\label{tb:pipelineprovenance}\\
%         \hline
%             \begin{itemize}
%                 \item
%                 {\small Design and code your pipeline as a chain of independently runnable modules (i.e. a module for feature extraction) 
%                 in a way that the output from each phase would be the input to the next one.}
                
%                 \item
%                 {\small Create a Jupyter Notebook (or any available alternative) to walk the reader through the pipeline 
%                 starting from the feature extraction all the way to result generation. for each step:}
%                     \begin{itemize}
%                         \item
%                         {\footnotesize Explain briefly what the code does (or make a reference to the correspondent part in the paper) 
%                         followed by a runnable code cell demonstrating how to run the code.}
%                        \item
%                         {\footnotesize If running the entire code for a part (i.e. feature selection) is not feasible, then leave the runnable 
%                         code cell there and explain why it is not feasible to run it through the notebook. Also, explain over the expected output 
%                         from this process (if available, you can load and display the process result instead)}
%                 \end{itemize}
%                 \item
%                 {\small If a random function is used anywhere throughout the whole process, then, 
%                 report on random function seed and the library details}
                
%                 \item
%                 {\small Report on any involved coded program, software or environment-related parameters, 
%                   according to section \ref{sec:softwareProvenance} guidelines. }
                
%                 \item
%                 {\small Version control and share the entire project (including the output from each phase)}
%                 \item
                
%                 {\small If possible, setup your pipeline on a container (or a virtual machine) and share the entire virtual environment in 
%                 the state that the results are produced.}
                
            
%             \end{itemize} &\\
%         \hline

%     \end{tabular}
%     \captionsetup{font=small,width=12cm}
%     \caption{Reproducible experiment report checklist (continued)}
%     \label{tab:table3}
    
% \end{table}