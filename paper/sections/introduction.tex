\section {Introduction}


Reproducibility, the ability to reproduce computational results using
identical data and software~\cite{peng2011reproducible}, is a cornerstone
of scientific methodology. In the past decade, however, several studies
revealed a widespread lack of results reproducibility, to the
point that the existence of a reproducibility crisis is now acknowledged in
various fields~\cite{baker2016reproducibility}.

Counter-measures were identified to improve results
reproducibility~\cite{sandve2013ten}, among which (1) study
pre-registrations~\cite{chambers2015registered}, to limit  p-hacking, (2)
open data and software sharing~\cite{wilkinson2016fair}, to reduce
technical and non-technical barriers to reproducibility studies, (3)
stricter reporting of computational experiments, including through Jupyter
notebooks~\cite{xxx}, checklists~\cite{nichols2017best} or structured
provenance formats~\cite{goble2020fair}, to provide a comprehensive documentation of
experiment conditions. The latter, in particular, is critical given the
important methodological flexibility associated with computational
experiments, exemplified in studies such as~\cite{carp2012plurality} where the number of
valid pipelines to process a dataset was found to be in the thousands.

Machine Learning experiments are not immune to reproducibility
issues~\cite{raff2019step}, in particular given the flexibility available in
data pre-processing, train/test set definitions, algorithm selection and
parametrization, and evaluation metrics. Due to the widespreading use of
Machine Learning in scientific fields, this has implications in various
disciplines such as in bioinformatics, our primary concern in this study.

This paper presents a reproducibility study of trans-membrane protein
classification using Machine Learning techniques. We report on our attempts
to reproduce a classical paper of the field~\cite{mishra2014prediction}, showing the impact
of methodological flexibility on classification performance, and
highlighting associated best practices to report Machine Learning results
for similar problems. 

Trans-membrane protein classification is an important problem in
bioinformatics. The selected paper is a well-cited, reference contribution
that we selected given the availability of its input data, the quality of
its writing and methods reporting, and its overall impact in the field. 
